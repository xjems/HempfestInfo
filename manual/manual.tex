%% Based on a TeXnicCenter-Template by Tino Weinkauf.
%%%%%%%%%%%%%%%%%%%%%%%%%%%%%%%%%%%%%%%%%%%%%%%%%%%%%%%%%%%%%

%%%%%%%%%%%%%%%%%%%%%%%%%%%%%%%%%%%%%%%%%%%%%%%%%%%%%%%%%%%%%
%% HEADER
%%%%%%%%%%%%%%%%%%%%%%%%%%%%%%%%%%%%%%%%%%%%%%%%%%%%%%%%%%%%%
\documentclass[a4paper,twoside,10pt]{book}
% Alternative Options:
%	Paper Size: a4paper / a5paper / b5paper / letterpaper / legalpaper / executivepaper
% Duplex: oneside / twoside
% Base Font Size: 10pt / 11pt / 12pt


%% Language %%%%%%%%%%%%%%%%%%%%%%%%%%%%%%%%%%%%%%%%%%%%%%%%%
\usepackage[USenglish]{babel} %francais, polish, spanish, ...
\usepackage[T1]{fontenc}
\usepackage[ansinew]{inputenc}

\usepackage{lmodern} %Type1-font for non-english texts and characters


%% Packages for Graphics & Figures %%%%%%%%%%%%%%%%%%%%%%%%%%
\usepackage{graphicx} %%For loading graphic files
%\usepackage{subfig} %%Subfigures inside a figure
%\usepackage{tikz} %%Generate vector graphics from within LaTeX

%% Please note:
%% Images can be included using \includegraphics{filename}
%% resp. using the dialog in the Insert menu.
%% 
%% The mode "LaTeX => PDF" allows the following formats:
%%   .jpg  .png  .pdf  .mps
%% 
%% The modes "LaTeX => DVI", "LaTeX => PS" und "LaTeX => PS => PDF"
%% allow the following formats:
%%   .eps  .ps  .bmp  .pict  .pntg


%% Math Packages %%%%%%%%%%%%%%%%%%%%%%%%%%%%%%%%%%%%%%%%%%%%
\usepackage{amsmath}
\usepackage{amsthm}
\usepackage{amsfonts}


%% Line Spacing %%%%%%%%%%%%%%%%%%%%%%%%%%%%%%%%%%%%%%%%%%%%%
%\usepackage{setspace}
%\singlespacing        %% 1-spacing (default)
%\onehalfspacing       %% 1,5-spacing
%\doublespacing        %% 2-spacing


%% Other Packages %%%%%%%%%%%%%%%%%%%%%%%%%%%%%%%%%%%%%%%%%%%
%\usepackage{a4wide} %%Smaller margins = more text per page.
%\usepackage{fancyhdr} %%Fancy headings
%\usepackage{longtable} %%For tables, that exceed one page


%%%%%%%%%%%%%%%%%%%%%%%%%%%%%%%%%%%%%%%%%%%%%%%%%%%%%%%%%%%%%
%% Remarks
%%%%%%%%%%%%%%%%%%%%%%%%%%%%%%%%%%%%%%%%%%%%%%%%%%%%%%%%%%%%%
%
% TODO:
% 1. Edit the used packages and their options (see above).
% 2. If you want, add a BibTeX-File to the project
%    (e.g., 'literature.bib').
% 3. Happy TeXing!
%
%%%%%%%%%%%%%%%%%%%%%%%%%%%%%%%%%%%%%%%%%%%%%%%%%%%%%%%%%%%%%

%%%%%%%%%%%%%%%%%%%%%%%%%%%%%%%%%%%%%%%%%%%%%%%%%%%%%%%%%%%%%
%% Options / Modifications
%%%%%%%%%%%%%%%%%%%%%%%%%%%%%%%%%%%%%%%%%%%%%%%%%%%%%%%%%%%%%

%\input{options} %You need a file 'options.tex' for this
%% ==> TeXnicCenter supplies some possible option files
%% ==> with its templates (File | New from Template...).



%%%%%%%%%%%%%%%%%%%%%%%%%%%%%%%%%%%%%%%%%%%%%%%%%%%%%%%%%%%%%
%% DOCUMENT
%%%%%%%%%%%%%%%%%%%%%%%%%%%%%%%%%%%%%%%%%%%%%%%%%%%%%%%%%%%%%
\begin{document}

\pagestyle{empty} %No headings for the first pages.


%% Title Page %%%%%%%%%%%%%%%%%%%%%%%%%%%%%%%%%%%%%%%%%%%%%%%
%% ==> Write your text here or include other files.

%% The simple version:
\title{Hempfest Info Manual}
\author{Talisha Lopez, Hempfest InfoQueen}
%\date{} %%If commented, the current date is used.
\maketitle

%% The nice version:
%\input{titlepage} %%You need a file 'titlepage.tex' for this.
%% ==> TeXnicCenter supplies a possible titlepage file
%% ==> with its templates (File | New from Template...).


%% Inhaltsverzeichnis %%%%%%%%%%%%%%%%%%%%%%%%%%%%%%%%%%%%%%%
\tableofcontents %Table of contents
\cleardoublepage %The first chapter should start on an odd page.

\pagestyle{plain} %Now display headings: headings / fancy / ...

\chapter{Frequently Asked Questions}

%Each question is will be a \section{}

\section{I'm a patient. Is there a protected place for me to medicate?}

Yes! The medical tent is located at the north end of stone village.

\section{I'm hungry! Where is the food?} %make sure this is updated at next core meeting!
 Food is located At:
\begin{itemize}
	\item Munchie Market, 206-218
	\item Northern Bites, 225-228
	\item Ganja Gardens, 219-224
	\item Grubbin' Grove, 200-204
	\item Bud Of Seattle
\end{itemize}

\section{I'm thirsty! Where is the water?}
\begin{itemize}
	\item Closest water sales booth is located Just behind us.
	\item No, I meant free water: H20 station next to ice truck on Blueberry Hill..
\end{itemize}


\section{Where is the smoking going on? Is it legal to smoke at Hempfest?}
Unfortunately, Marijauna is still illegal. Penalties are increased in city parks, as they are considered drug free zones. If you choose to partake as a form of civil disobedience, you do so at your own risk. 

\section{Is Seattle Hempfest the biggest pot rally in the world?}
Yes, as far as we can tell!

\section{Where can I find information about Medical Marijauna?}
Stone Village.

\section{When does Hempfest close?}
Hempfest closes at 8 PM. 

\section{Where can I get official Hempfest swag?}
\begin{itemize}
	\item 420 Store, Just west of us, between 403 and 404.
	\item Hempfest General Store, Just south of McWilliams Stage.
\end{itemize}

\section{How do I get a staff tshirt? Can I buy yours?}
Sign up to volunteer at Staff Check in. No, you can only get a shirt if  you volunteer.

\section{Where are the ATMs??}
\begin{itemize}
	\item West side of hemposium
	\item Just around corner from info booth, to the north
	\item Just north of booth 529
	\item Betwen McWilliams and Ganja Gardens
	\item South of Dancesafe, on Hwy 420
\end{itemize}

\section{Is it safe to buy pot brownies at Hempfest?}
No. They are illegal. They are also unpermitted, unregulated food vending and can be dangerous to your health. If you see someone selling brownies, please report to any HF staff with a radio or to HF security. 

\section{Can I have a program?}
Yes. We are asking for a four dollar donation per program, unless you have already donated, then its one dollar. 

\section{Where can I get a beer?}
There is NO alcohol at Hempfest. 

\section{When is the free joint toss?}
What joints? Marijauna is illegal! 

\section{I heard a rumor Snoop Dogg/anyone else not on the schedule is playing, is that true?}
No, all the acts playing are in the schedule.

\section{Where are the closest bathrooms?}
Directly to the east on hwy 420.

%\section{What do I do if the police approach me?}
%need to add actual content here, possible to get lit from somewhere? NORML?
 
\section{Can I camp here overnight?}
NO, Camping is illegal in Seattle parks.

\section{Thats some map! How long is the park?}
1.5 miles! 

\section{I need a band aid, I'm having a panic attack, I think I'm dying, where do I go?}
First aid is located right next to the info booth. (If necessary, escort to booth!)









\chapter{General Info}

\Large\begin{quotation}
Hempfest is an Action Movie�Not a Drama. 


\begin{center}
-V. McPeak
\end{center}
\end{quotation}

\section{Overview of Information Booth}
Hempfest Information is one the main public faces of Hempfest org at the event. We have a responsibility to provide timely and accurate information to all festival goers who visit us,along with a smile! The booth is located just north of james junction, south of munchie market. 

\section{Hempfest Mission Statement}
To educate the public on the myriad of potential benefits offered by the Cannabis plant, including the medicinal, industrial, agricultural, economic, environmental, and other benefits and applications. In particular, Seattle Hempfest seeks to advance the cause of Cannabis policy reform through education, while advancing the public image of the Cannabis advocate or enthusiast through example.


\section{What is Hempfest about?}
Human rights. Equality. Freedom. We believe that adults in a free society deserve the right to make their own educated and informed choices about what they put into their own bodies. We believe that those important health choices should be made based upon truthful and accurate information. Hempfest is a political rally and demonstration against America's laws criminalizing and imprisoning people who use cannabis. Hempfest is also about cannabis; industrial hemp, medical marijuana and recreational use by otherwise law abiding, responsible adults. We see Hempfest as about freedom first, and the cannabis plant second.


\section{Why is Hempfest free?}
We would never do anything that might potentially prevent someone from coming and hearing our message of freedom, liberty and responsibility. By being free, Hempfest retains its status as a constitutionally protected free speech event, and allows everyone to come and participate, regardless of income.

\section{Park rules}
No: Pets (Dogs, Cats, Birds, etc.), Alcohol, Narcotics, Weapons, Camping, or Unauthorized Vending.

\section{Found Kids and Lost Parents}
\begin{itemize}
	\item 	 If you find a lost child or a parent who has lost a child, stay with the parent or kid and send someone else to find someone with a radio! Usually the person that they�re looking for re-appears at this point. If not, when the radio arrives report the child�s:
\begin{itemize}
	\item Name
\item Age
\item Description, including clothes
\item Last location seen
\end{itemize}
\item Wait for further instructions from Operations. They might tell you to stay put but in general lost kids are taken to the Police, located at the Pumphouse, or the closest SPD officer.
\item If the person that they�re looking for shows up before this cycle is complete, please report that the situation is clear to Operations � otherwise we�ll all be looking for this kid forever. This is a service that we reserve for parents and small children. We don�t have time to help people find their friends. However, if there appear to be extenuating circumstances or if a person appears to be highly distressed please contact Operations and request assistance.
\end{itemize}

\section{Unauthorized vending}
ALL  vendors must have a PASS. If someone asks about being a blanket vendor and they do not already have a pass, then we need to send them to vending to get one. 

\section{Code of Conduct}
All Info Booth staff MUST abide by the Hempfest Code of Conduct. PLEASE do not do anything illegal in your staff shirt! 

\section{Keeping track of Info Booth visitors}
If you are assigned to keep track of visitors to the info booth, please click only ONCE for each visitor. If someone asks a particular interesting question or you get a repeated question not on the FAQ list, please let the info booth coordinator know! 

\section{Finding Specific Vendors}
This is arguably THE most important job of the information booth. Hempfest relies on our vendors! Our goal is to send every person who asks to the vendor they want to visit! 

\section {How to find vendors}
\begin{itemize}
	\item Use list to find vendor name
	\item Get booth number
	\item Locate on map, give directions to patron
	\item Don't forget to tell them, Hempfest Info sent you!
\end{itemize}

\section{Conflict Resolution}

\section{What Hempfest offers to visitors}

\section{Hemposium, and why its important}
Hempfest features an area and stage called the Hemposium that features exhibits, displays, demonstrations, panel discussions and featured speakers on the issues that include industrial hemp and its many uses. Hemposium is located just south of the staff compound. If a patron asks where they can find more information about Hemp and its many uses, please send them to Hemposium! 

\section{Asking for donations}
Hempfest is a non profit organization and produced by volunteers. Please consider asking all visitors to the information booth to drop a dollar or two into our donation bucket. It all goes back into the fest! We need the public's help to continue our message of freedom! 

\section{Voter Registration}

\section{Medical MJ Information}
% www.cdc.coop/resources

\section{Don't just burn it, learn it! How we can further educate the public}

\section{Resources: Laws, cannabis reform, constitutional rights}

\section{Hempfest and the environment, what we are doing to be more responsible}



%\chapter{Lost and Found Procedures}





%%%%%%%%%%%%%%%%%%%%%%%%%%%%%%%%%%%%%%%%%%%%%%%%%%%%%%%%%%%%%
%% APPENDICES
%%%%%%%%%%%%%%%%%%%%%%%%%%%%%%%%%%%%%%%%%%%%%%%%%%%%%%%%%%%%%
\appendix
%% ==> Write your text here or include other files.

%\input{FileName} %You need a file 'FileName.tex' for this.


\end{document}

